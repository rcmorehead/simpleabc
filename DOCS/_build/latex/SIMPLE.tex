% Generated by Sphinx.
\def\sphinxdocclass{report}
\documentclass[letterpaper,10pt,english]{sphinxmanual}
\usepackage[utf8]{inputenc}
\DeclareUnicodeCharacter{00A0}{\nobreakspace}
\usepackage{cmap}
\usepackage[T1]{fontenc}
\usepackage{babel}
\usepackage{times}
\usepackage[Bjarne]{fncychap}
\usepackage{longtable}
\usepackage{sphinx}
\usepackage{multirow}


\title{SIMPLE Documentation}
\date{March 19, 2014}
\release{0.1.0}
\author{Robert C. Morehead}
\newcommand{\sphinxlogo}{}
\renewcommand{\releasename}{Release}
\makeindex

\makeatletter
\def\PYG@reset{\let\PYG@it=\relax \let\PYG@bf=\relax%
    \let\PYG@ul=\relax \let\PYG@tc=\relax%
    \let\PYG@bc=\relax \let\PYG@ff=\relax}
\def\PYG@tok#1{\csname PYG@tok@#1\endcsname}
\def\PYG@toks#1+{\ifx\relax#1\empty\else%
    \PYG@tok{#1}\expandafter\PYG@toks\fi}
\def\PYG@do#1{\PYG@bc{\PYG@tc{\PYG@ul{%
    \PYG@it{\PYG@bf{\PYG@ff{#1}}}}}}}
\def\PYG#1#2{\PYG@reset\PYG@toks#1+\relax+\PYG@do{#2}}

\expandafter\def\csname PYG@tok@gd\endcsname{\def\PYG@tc##1{\textcolor[rgb]{0.63,0.00,0.00}{##1}}}
\expandafter\def\csname PYG@tok@gu\endcsname{\let\PYG@bf=\textbf\def\PYG@tc##1{\textcolor[rgb]{0.50,0.00,0.50}{##1}}}
\expandafter\def\csname PYG@tok@gt\endcsname{\def\PYG@tc##1{\textcolor[rgb]{0.00,0.27,0.87}{##1}}}
\expandafter\def\csname PYG@tok@gs\endcsname{\let\PYG@bf=\textbf}
\expandafter\def\csname PYG@tok@gr\endcsname{\def\PYG@tc##1{\textcolor[rgb]{1.00,0.00,0.00}{##1}}}
\expandafter\def\csname PYG@tok@cm\endcsname{\let\PYG@it=\textit\def\PYG@tc##1{\textcolor[rgb]{0.25,0.50,0.56}{##1}}}
\expandafter\def\csname PYG@tok@vg\endcsname{\def\PYG@tc##1{\textcolor[rgb]{0.73,0.38,0.84}{##1}}}
\expandafter\def\csname PYG@tok@m\endcsname{\def\PYG@tc##1{\textcolor[rgb]{0.13,0.50,0.31}{##1}}}
\expandafter\def\csname PYG@tok@mh\endcsname{\def\PYG@tc##1{\textcolor[rgb]{0.13,0.50,0.31}{##1}}}
\expandafter\def\csname PYG@tok@cs\endcsname{\def\PYG@tc##1{\textcolor[rgb]{0.25,0.50,0.56}{##1}}\def\PYG@bc##1{\setlength{\fboxsep}{0pt}\colorbox[rgb]{1.00,0.94,0.94}{\strut ##1}}}
\expandafter\def\csname PYG@tok@ge\endcsname{\let\PYG@it=\textit}
\expandafter\def\csname PYG@tok@vc\endcsname{\def\PYG@tc##1{\textcolor[rgb]{0.73,0.38,0.84}{##1}}}
\expandafter\def\csname PYG@tok@il\endcsname{\def\PYG@tc##1{\textcolor[rgb]{0.13,0.50,0.31}{##1}}}
\expandafter\def\csname PYG@tok@go\endcsname{\def\PYG@tc##1{\textcolor[rgb]{0.20,0.20,0.20}{##1}}}
\expandafter\def\csname PYG@tok@cp\endcsname{\def\PYG@tc##1{\textcolor[rgb]{0.00,0.44,0.13}{##1}}}
\expandafter\def\csname PYG@tok@gi\endcsname{\def\PYG@tc##1{\textcolor[rgb]{0.00,0.63,0.00}{##1}}}
\expandafter\def\csname PYG@tok@gh\endcsname{\let\PYG@bf=\textbf\def\PYG@tc##1{\textcolor[rgb]{0.00,0.00,0.50}{##1}}}
\expandafter\def\csname PYG@tok@ni\endcsname{\let\PYG@bf=\textbf\def\PYG@tc##1{\textcolor[rgb]{0.84,0.33,0.22}{##1}}}
\expandafter\def\csname PYG@tok@nl\endcsname{\let\PYG@bf=\textbf\def\PYG@tc##1{\textcolor[rgb]{0.00,0.13,0.44}{##1}}}
\expandafter\def\csname PYG@tok@nn\endcsname{\let\PYG@bf=\textbf\def\PYG@tc##1{\textcolor[rgb]{0.05,0.52,0.71}{##1}}}
\expandafter\def\csname PYG@tok@no\endcsname{\def\PYG@tc##1{\textcolor[rgb]{0.38,0.68,0.84}{##1}}}
\expandafter\def\csname PYG@tok@na\endcsname{\def\PYG@tc##1{\textcolor[rgb]{0.25,0.44,0.63}{##1}}}
\expandafter\def\csname PYG@tok@nb\endcsname{\def\PYG@tc##1{\textcolor[rgb]{0.00,0.44,0.13}{##1}}}
\expandafter\def\csname PYG@tok@nc\endcsname{\let\PYG@bf=\textbf\def\PYG@tc##1{\textcolor[rgb]{0.05,0.52,0.71}{##1}}}
\expandafter\def\csname PYG@tok@nd\endcsname{\let\PYG@bf=\textbf\def\PYG@tc##1{\textcolor[rgb]{0.33,0.33,0.33}{##1}}}
\expandafter\def\csname PYG@tok@ne\endcsname{\def\PYG@tc##1{\textcolor[rgb]{0.00,0.44,0.13}{##1}}}
\expandafter\def\csname PYG@tok@nf\endcsname{\def\PYG@tc##1{\textcolor[rgb]{0.02,0.16,0.49}{##1}}}
\expandafter\def\csname PYG@tok@si\endcsname{\let\PYG@it=\textit\def\PYG@tc##1{\textcolor[rgb]{0.44,0.63,0.82}{##1}}}
\expandafter\def\csname PYG@tok@s2\endcsname{\def\PYG@tc##1{\textcolor[rgb]{0.25,0.44,0.63}{##1}}}
\expandafter\def\csname PYG@tok@vi\endcsname{\def\PYG@tc##1{\textcolor[rgb]{0.73,0.38,0.84}{##1}}}
\expandafter\def\csname PYG@tok@nt\endcsname{\let\PYG@bf=\textbf\def\PYG@tc##1{\textcolor[rgb]{0.02,0.16,0.45}{##1}}}
\expandafter\def\csname PYG@tok@nv\endcsname{\def\PYG@tc##1{\textcolor[rgb]{0.73,0.38,0.84}{##1}}}
\expandafter\def\csname PYG@tok@s1\endcsname{\def\PYG@tc##1{\textcolor[rgb]{0.25,0.44,0.63}{##1}}}
\expandafter\def\csname PYG@tok@gp\endcsname{\let\PYG@bf=\textbf\def\PYG@tc##1{\textcolor[rgb]{0.78,0.36,0.04}{##1}}}
\expandafter\def\csname PYG@tok@sh\endcsname{\def\PYG@tc##1{\textcolor[rgb]{0.25,0.44,0.63}{##1}}}
\expandafter\def\csname PYG@tok@ow\endcsname{\let\PYG@bf=\textbf\def\PYG@tc##1{\textcolor[rgb]{0.00,0.44,0.13}{##1}}}
\expandafter\def\csname PYG@tok@sx\endcsname{\def\PYG@tc##1{\textcolor[rgb]{0.78,0.36,0.04}{##1}}}
\expandafter\def\csname PYG@tok@bp\endcsname{\def\PYG@tc##1{\textcolor[rgb]{0.00,0.44,0.13}{##1}}}
\expandafter\def\csname PYG@tok@c1\endcsname{\let\PYG@it=\textit\def\PYG@tc##1{\textcolor[rgb]{0.25,0.50,0.56}{##1}}}
\expandafter\def\csname PYG@tok@kc\endcsname{\let\PYG@bf=\textbf\def\PYG@tc##1{\textcolor[rgb]{0.00,0.44,0.13}{##1}}}
\expandafter\def\csname PYG@tok@c\endcsname{\let\PYG@it=\textit\def\PYG@tc##1{\textcolor[rgb]{0.25,0.50,0.56}{##1}}}
\expandafter\def\csname PYG@tok@mf\endcsname{\def\PYG@tc##1{\textcolor[rgb]{0.13,0.50,0.31}{##1}}}
\expandafter\def\csname PYG@tok@err\endcsname{\def\PYG@bc##1{\setlength{\fboxsep}{0pt}\fcolorbox[rgb]{1.00,0.00,0.00}{1,1,1}{\strut ##1}}}
\expandafter\def\csname PYG@tok@kd\endcsname{\let\PYG@bf=\textbf\def\PYG@tc##1{\textcolor[rgb]{0.00,0.44,0.13}{##1}}}
\expandafter\def\csname PYG@tok@ss\endcsname{\def\PYG@tc##1{\textcolor[rgb]{0.32,0.47,0.09}{##1}}}
\expandafter\def\csname PYG@tok@sr\endcsname{\def\PYG@tc##1{\textcolor[rgb]{0.14,0.33,0.53}{##1}}}
\expandafter\def\csname PYG@tok@mo\endcsname{\def\PYG@tc##1{\textcolor[rgb]{0.13,0.50,0.31}{##1}}}
\expandafter\def\csname PYG@tok@mi\endcsname{\def\PYG@tc##1{\textcolor[rgb]{0.13,0.50,0.31}{##1}}}
\expandafter\def\csname PYG@tok@kn\endcsname{\let\PYG@bf=\textbf\def\PYG@tc##1{\textcolor[rgb]{0.00,0.44,0.13}{##1}}}
\expandafter\def\csname PYG@tok@o\endcsname{\def\PYG@tc##1{\textcolor[rgb]{0.40,0.40,0.40}{##1}}}
\expandafter\def\csname PYG@tok@kr\endcsname{\let\PYG@bf=\textbf\def\PYG@tc##1{\textcolor[rgb]{0.00,0.44,0.13}{##1}}}
\expandafter\def\csname PYG@tok@s\endcsname{\def\PYG@tc##1{\textcolor[rgb]{0.25,0.44,0.63}{##1}}}
\expandafter\def\csname PYG@tok@kp\endcsname{\def\PYG@tc##1{\textcolor[rgb]{0.00,0.44,0.13}{##1}}}
\expandafter\def\csname PYG@tok@w\endcsname{\def\PYG@tc##1{\textcolor[rgb]{0.73,0.73,0.73}{##1}}}
\expandafter\def\csname PYG@tok@kt\endcsname{\def\PYG@tc##1{\textcolor[rgb]{0.56,0.13,0.00}{##1}}}
\expandafter\def\csname PYG@tok@sc\endcsname{\def\PYG@tc##1{\textcolor[rgb]{0.25,0.44,0.63}{##1}}}
\expandafter\def\csname PYG@tok@sb\endcsname{\def\PYG@tc##1{\textcolor[rgb]{0.25,0.44,0.63}{##1}}}
\expandafter\def\csname PYG@tok@k\endcsname{\let\PYG@bf=\textbf\def\PYG@tc##1{\textcolor[rgb]{0.00,0.44,0.13}{##1}}}
\expandafter\def\csname PYG@tok@se\endcsname{\let\PYG@bf=\textbf\def\PYG@tc##1{\textcolor[rgb]{0.25,0.44,0.63}{##1}}}
\expandafter\def\csname PYG@tok@sd\endcsname{\let\PYG@it=\textit\def\PYG@tc##1{\textcolor[rgb]{0.25,0.44,0.63}{##1}}}

\def\PYGZbs{\char`\\}
\def\PYGZus{\char`\_}
\def\PYGZob{\char`\{}
\def\PYGZcb{\char`\}}
\def\PYGZca{\char`\^}
\def\PYGZam{\char`\&}
\def\PYGZlt{\char`\<}
\def\PYGZgt{\char`\>}
\def\PYGZsh{\char`\#}
\def\PYGZpc{\char`\%}
\def\PYGZdl{\char`\$}
\def\PYGZhy{\char`\-}
\def\PYGZsq{\char`\'}
\def\PYGZdq{\char`\"}
\def\PYGZti{\char`\~}
% for compatibility with earlier versions
\def\PYGZat{@}
\def\PYGZlb{[}
\def\PYGZrb{]}
\makeatother

\begin{document}

\maketitle
\tableofcontents
\phantomsection\label{index::doc}


Contents:
\phantomsection\label{index:module-simple_lib}\index{simple\_lib (module)}
Useful classes and functions for SIMPLE.
\index{impact\_parameter() (in module simple\_lib)}

\begin{fulllineitems}
\phantomsection\label{index:simple_lib.impact_parameter}\pysiglinewithargsret{\code{simple\_lib.}\bfcode{impact\_parameter}}{\emph{a}, \emph{e}, \emph{i}, \emph{w}, \emph{r\_star}}{}
Compute the impact parameter at for a transiting planet.
\begin{quote}\begin{description}
\item[{Parameters}] \leavevmode
\textbf{a} : int, float or numpy array
\begin{quote}

Semimajor axis of planet's orbit in AU
\end{quote}

\textbf{e} : int, float or numpy array
\begin{quote}

Eccentricity of planet. WARNING! This function breaks down at
high eccentricity (\textgreater{}\textgreater{} 0.9), so be careful!
\end{quote}

\textbf{i} : int, float or numpy array
\begin{quote}

Inclination of planet in degrees. 90 degrees is edge-on.
\end{quote}

\textbf{w} : int, float or numpy array
\begin{quote}

Longitude of ascending node defined with respect to sky-plane.
\end{quote}

\textbf{r\_star} : int, float or numpy array
\begin{quote}

Radius of star in solar radii.
\end{quote}

\item[{Returns}] \leavevmode
\textbf{b} : float or numpy array
\begin{quote}

The impact parameter, ie transit latitude in units of stellar radius.
\end{quote}

\end{description}\end{quote}
\paragraph{Notes}

Using Eqn. (7), Chap. 4, Page  56 of Exoplanets, edited by S. Seager.
Tucson, AZ: University of Arizona Press, 2011, 526 pp.
ISBN 978-0-8165-2945-2.
\paragraph{Examples}

\begin{Verbatim}[commandchars=\\\{\}]
\PYG{g+gp}{\PYGZgt{}\PYGZgt{}\PYGZgt{} }\PYG{n}{impact\PYGZus{}parameter}\PYG{p}{(}\PYG{l+m+mi}{1}\PYG{p}{,} \PYG{l+m+mi}{0}\PYG{p}{,} \PYG{l+m+mi}{90}\PYG{p}{,} \PYG{l+m+mi}{0}\PYG{p}{,} \PYG{l+m+mi}{1}\PYG{p}{)}
\PYG{g+go}{1.3171077641937547e\PYGZhy{}14}
\PYG{g+gp}{\PYGZgt{}\PYGZgt{}\PYGZgt{} }\PYG{n}{a} \PYG{o}{=} \PYG{n}{np}\PYG{o}{.}\PYG{n}{linspace}\PYG{p}{(}\PYG{o}{.}\PYG{l+m+mi}{1}\PYG{p}{,} \PYG{l+m+mf}{1.5}\PYG{p}{,} \PYG{l+m+mi}{3}\PYG{p}{)}
\PYG{g+gp}{\PYGZgt{}\PYGZgt{}\PYGZgt{} }\PYG{n}{e} \PYG{o}{=} \PYG{n}{np}\PYG{o}{.}\PYG{n}{linspace}\PYG{p}{(}\PYG{l+m+mi}{0}\PYG{p}{,} \PYG{o}{.}\PYG{l+m+mi}{9}\PYG{p}{,} \PYG{l+m+mi}{3}\PYG{p}{)}
\PYG{g+gp}{\PYGZgt{}\PYGZgt{}\PYGZgt{} }\PYG{n}{i} \PYG{o}{=} \PYG{n}{np}\PYG{o}{.}\PYG{n}{linspace}\PYG{p}{(}\PYG{l+m+mi}{89}\PYG{p}{,} \PYG{l+m+mi}{91}\PYG{p}{,} \PYG{l+m+mi}{3}\PYG{p}{)}
\PYG{g+gp}{\PYGZgt{}\PYGZgt{}\PYGZgt{} }\PYG{n}{w} \PYG{o}{=} \PYG{n}{np}\PYG{o}{.}\PYG{n}{linspace}\PYG{p}{(}\PYG{l+m+mi}{0}\PYG{p}{,} \PYG{l+m+mi}{360}\PYG{p}{,} \PYG{l+m+mi}{3}\PYG{p}{)}
\PYG{g+gp}{\PYGZgt{}\PYGZgt{}\PYGZgt{} }\PYG{n}{r\PYGZus{}star} \PYG{o}{=} \PYG{n}{np}\PYG{o}{.}\PYG{n}{linspace}\PYG{p}{(}\PYG{l+m+mf}{0.1}\PYG{p}{,} \PYG{l+m+mi}{10}\PYG{p}{,} \PYG{l+m+mi}{3}\PYG{p}{)}
\PYG{g+gp}{\PYGZgt{}\PYGZgt{}\PYGZgt{} }\PYG{n}{impact\PYGZus{}parameter}\PYG{p}{(}\PYG{n}{a}\PYG{p}{,} \PYG{n}{e}\PYG{p}{,} \PYG{n}{i}\PYG{p}{,} \PYG{n}{w}\PYG{p}{,} \PYG{n}{r\PYGZus{}star}\PYG{p}{)}
\PYG{g+go}{array([  3.75401300e+00,   1.66398961e\PYGZhy{}15,   1.06989371e\PYGZhy{}01])}
\end{Verbatim}

\end{fulllineitems}

\index{inclination() (in module simple\_lib)}

\begin{fulllineitems}
\phantomsection\label{index:simple_lib.inclination}\pysiglinewithargsret{\code{simple\_lib.}\bfcode{inclination}}{\emph{fund\_plane}, \emph{mutual\_inc}, \emph{node}}{}
Compute the inclination of a planet.

Uses the law a spherical cosines to compute the sky plane of a orbit
given a reference plane inclination, angle from reference plane (ie mutual
inclination) and a nodal angle.
\begin{quote}\begin{description}
\item[{Parameters}] \leavevmode
\textbf{fund\_plane: int, float or numpy array} :
\begin{quote}

Inclination of of the fundamental plane of the system in degrees with
respect to the sky plane 90 degrees is edge-on.
\end{quote}

\textbf{mutual\_inc} : int, float or numpy array
\begin{quote}

Angle in degrees of the orbital plane of the planet with respect to the
fundamental plane of the system.
\end{quote}

\textbf{node} : int, float or numpy array
\begin{quote}

Rotation in degrees of the planet's orbit about the perpendicular of
the reference plane. I.e. the longitude of the node with respect to the
reference plane.
\end{quote}

\item[{Returns}] \leavevmode
\textbf{i} : float or numpy array
\begin{quote}

The inclination of the planet's orbit with respect to the sky plane.
\end{quote}

\end{description}\end{quote}
\paragraph{Notes}

See eqn. () in
\paragraph{Examples}

\begin{Verbatim}[commandchars=\\\{\}]
\PYG{g+gp}{\PYGZgt{}\PYGZgt{}\PYGZgt{} }\PYG{n}{inclination}\PYG{p}{(}\PYG{l+m+mi}{90}\PYG{p}{,} \PYG{l+m+mi}{3}\PYG{p}{,} \PYG{l+m+mi}{0}\PYG{p}{)}
\PYG{g+go}{87.0}
\PYG{g+gp}{\PYGZgt{}\PYGZgt{}\PYGZgt{} }\PYG{n}{fun\PYGZus{}i} \PYG{o}{=} \PYG{n}{np}\PYG{o}{.}\PYG{n}{linspace}\PYG{p}{(}\PYG{l+m+mi}{80}\PYG{p}{,} \PYG{l+m+mi}{110}\PYG{p}{,} \PYG{l+m+mi}{3}\PYG{p}{)}
\PYG{g+gp}{\PYGZgt{}\PYGZgt{}\PYGZgt{} }\PYG{n}{mi} \PYG{o}{=} \PYG{n}{np}\PYG{o}{.}\PYG{n}{linspace}\PYG{p}{(}\PYG{l+m+mi}{0}\PYG{p}{,} \PYG{l+m+mi}{10}\PYG{p}{,} \PYG{l+m+mi}{3}\PYG{p}{)}
\PYG{g+gp}{\PYGZgt{}\PYGZgt{}\PYGZgt{} }\PYG{n}{node} \PYG{o}{=} \PYG{n}{np}\PYG{o}{.}\PYG{n}{linspace}\PYG{p}{(}\PYG{l+m+mi}{30}\PYG{p}{,}\PYG{l+m+mi}{100}\PYG{p}{,}\PYG{l+m+mi}{3}\PYG{p}{)}
\PYG{g+gp}{\PYGZgt{}\PYGZgt{}\PYGZgt{} }\PYG{n}{inclination}\PYG{p}{(}\PYG{n}{fun\PYGZus{}i}\PYG{p}{,} \PYG{n}{mi}\PYG{p}{,} \PYG{n}{node}\PYG{p}{)}
\PYG{g+go}{array([  80.        ,   92.87347869,  111.41738591])}
\end{Verbatim}

\end{fulllineitems}

\index{semimajor\_axis() (in module simple\_lib)}

\begin{fulllineitems}
\phantomsection\label{index:simple_lib.semimajor_axis}\pysiglinewithargsret{\code{simple\_lib.}\bfcode{semimajor\_axis}}{\emph{period}, \emph{mass}}{}
Compute the semimajor axis of an object.

This is a simple implementation of the general form Kepler's Third law.
\begin{quote}\begin{description}
\item[{Parameters}] \leavevmode
\textbf{period} : int, float or numpy array
\begin{quote}

The orbital period of the orbiting body in units of days.
\end{quote}

\textbf{mass} : int, float or array-like
\begin{quote}

The mass of the central body (or mass sum) in units of solar mass.
\end{quote}

\item[{Returns}] \leavevmode
\textbf{a} : float or numpy array
\begin{quote}

The semimajor axis in AU.
\end{quote}

\end{description}\end{quote}
\paragraph{Examples}

\begin{Verbatim}[commandchars=\\\{\}]
\PYG{g+gp}{\PYGZgt{}\PYGZgt{}\PYGZgt{} }\PYG{n}{semimajor\PYGZus{}axis}\PYG{p}{(}\PYG{l+m+mf}{365.256363}\PYG{p}{,}\PYG{l+m+mf}{1.00}\PYG{p}{)}
\PYG{g+go}{0.999985270598628}
\end{Verbatim}

\begin{Verbatim}[commandchars=\\\{\}]
\PYG{g+gp}{\PYGZgt{}\PYGZgt{}\PYGZgt{} }\PYG{n}{semimajor\PYGZus{}axis}\PYG{p}{(}\PYG{n}{np}\PYG{o}{.}\PYG{n}{linspace}\PYG{p}{(}\PYG{l+m+mi}{1}\PYG{p}{,} \PYG{l+m+mi}{1000}\PYG{p}{,} \PYG{l+m+mi}{5}\PYG{p}{)}\PYG{p}{,}\PYG{n}{np}\PYG{o}{.}\PYG{n}{linspace}\PYG{p}{(}\PYG{l+m+mf}{0.08}\PYG{p}{,} \PYG{l+m+mi}{4}\PYG{p}{,} \PYG{l+m+mi}{5}\PYG{p}{)}\PYG{p}{)}
\PYG{g+go}{array([ 0.00843254,  0.7934587 ,  1.56461631,  2.33561574,  3.10657426])}
\end{Verbatim}

\end{fulllineitems}

\index{transit\_depth() (in module simple\_lib)}

\begin{fulllineitems}
\phantomsection\label{index:simple_lib.transit_depth}\pysiglinewithargsret{\code{simple\_lib.}\bfcode{transit\_depth}}{}{}
One-line description

Full description

\end{fulllineitems}

\index{transit\_duration() (in module simple\_lib)}

\begin{fulllineitems}
\phantomsection\label{index:simple_lib.transit_duration}\pysiglinewithargsret{\code{simple\_lib.}\bfcode{transit\_duration}}{\emph{p}, \emph{a}, \emph{e}, \emph{i}, \emph{w}, \emph{b}, \emph{r\_star}, \emph{r\_planet}}{}
Compute the full (Q1-Q4) transit duration.

Full description
\begin{quote}\begin{description}
\item[{Parameters}] \leavevmode
\textbf{p} : int, float or numpy array
\begin{quote}

Period of planet orbit in days
\end{quote}

\textbf{a} : int, float or numpy array
\begin{quote}

Semimajor axis of planet's orbit in AU
\end{quote}

\textbf{e} : int, float or numpy array
\begin{quote}

Eccentricity of planet. WARNING! This function breaks down at
high eccentricity (\textgreater{}\textgreater{} 0.9), so be careful!
\end{quote}

\textbf{i} : int, float or numpy array
\begin{quote}

Inclination of planet in degrees. 90 degrees is edge-on.
\end{quote}

\textbf{w} : int, float or numpy array
\begin{quote}

Longitude of ascending node defined with respect to sky-plane.
\end{quote}

\textbf{b} : int, float or numpy array
\begin{quote}

Impact parameter of planet.
\end{quote}

\textbf{r\_star} : int, float or numpy array
\begin{quote}

Radius of star in solar radii.
\end{quote}

\textbf{r\_planet} : int, float or numpy array
\begin{quote}

Radius of planet in Earth radii
\end{quote}

\item[{Returns}] \leavevmode
\textbf{T} : float or numpy array
\begin{quote}

The Q1-Q4 (full) transit duration of the planet in hours.
\end{quote}

\end{description}\end{quote}
\paragraph{Notes}

Using Eqns. (15) and (16), Chap. 4, Page  58 of Exoplanets, edited by S.
Seager. Tucson, AZ: University of Arizona Press, 2011, 526 pp.
ISBN 978-0-8165-2945-2.

\end{fulllineitems}



\chapter{Indices and tables}
\label{index:indices-and-tables}\label{index:welcome-to-simple-s-documentation}\begin{itemize}
\item {} 
\emph{genindex}

\item {} 
\emph{modindex}

\item {} 
\emph{search}

\end{itemize}



\renewcommand{\indexname}{Index}
\printindex
\end{document}
